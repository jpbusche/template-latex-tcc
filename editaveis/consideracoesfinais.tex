\chapter[Considerações Finais]{Considerações Finais}
No decorrer desta primeira parte, tive muita dificuldade em entender quais são as métricas e como iria montá-las, porém em conversas com o orientador e com membros de equipes de desenvolvimento de jogos, pude montar métricas que satisfaziam o escopo inicial do software. Na parte de extração foi possível a criação inicial do handler e de um plugin. Embora o que foi construído possui caráter básico, a primeira parte do projeto foi um sucesso.
\section{Trabalhos Futuros}
Em relação aos trabalhos futuros, podemos listar as seguintes atividades:
\begin{enumerate}
	\item \label{t1} Definição dos requisitos do sistema.
	\item \label{t2} Definição de quais dados serão extraídos.
	\item \label{t3} Definição das métricas do sistema.
	\item \label{t4} Estudo sobre game analytics e business intelligence.
	\item \label{t5} Implementação do protótipo da plataforma web.
	\item \label{t6} Escrita do TCC1.
	\item \label{t7} Implementação do handler e main do software de extração
	\item \label{t8} Implementação dos plugins do software de extração.
	\item \label{t9} Implementação da API de jogos.
	\item \label{t10} Implementação da API de rankings.
	\item \label{t11} Teste dos módulos da API.
	\item \label{t12} Implementação da  visualização das métricas.
	\item \label{t13} Implementação da visualização dos rankings.
	\item \label{t14} Implementação das sugestões da plataforma w
	\item \label{t15} Escrita do TCC2.
\end{enumerate}
O cronograma do trabalho está apresentado na tabela \ref{table:cronograma}.
\begin{table}
\centering
\begin{tabular}{|c|c|c|c|c|c|c|c|c|c|c|}
\hline \multicolumn{11}{|c|}{2018} \\
\hline &MAR&ABR&MAI&JUN&JUL&AGO&SET&OUT&NOV&DEZ \\
\hline \ref{t1} & \cellcolor{green}&&&&&&&&& \\
\hline \ref{t2} & \cellcolor{green}&\cellcolor{green}&&&&&&&& \\
\hline \ref{t3} && \cellcolor{green}&\cellcolor{green}&&&&&&& \\
\hline \ref{t4} && \cellcolor{green}&\cellcolor{green}&&&&&&& \\
\hline \ref{t5} &&& \cellcolor{green}&\cellcolor{green}&&&&&& \\
\hline \ref{t6} && \cellcolor{green}&\cellcolor{green}&\cellcolor{green}&&&&&& \\
\hline \ref{t7} &&&&& \cellcolor{yellow}&\cellcolor{yellow}&&&& \\
\hline \ref{t8} &&&&& \cellcolor{yellow}&\cellcolor{yellow}&&&& \\
\hline \ref{t9} &&&&&& \cellcolor{red}&\cellcolor{red}&&& \\
\hline \ref{t10} &&&&&& \cellcolor{red}&\cellcolor{red}&&& \\
\hline \ref{t11} &&&&&&& \cellcolor{red}&\cellcolor{red}&& \\
\hline \ref{t12} &&&&&&& \cellcolor{red}&\cellcolor{red}&\cellcolor{red}& \\
\hline \ref{t13} &&&&&&& \cellcolor{red}&\cellcolor{red}&\cellcolor{red}& \\
\hline \ref{t14} &&&&&&&& \cellcolor{red}&\cellcolor{red}&\cellcolor{red} \\
\hline \ref{t15} &&&&&&& \cellcolor{red}&\cellcolor{red}&\cellcolor{red}&\cellcolor{red} \\
\hline \multicolumn{11}{|c|}{Legendas} \\
\hline \multicolumn{10}{|l|}{Tarefas realizadas}&\cellcolor{green} \\
\hline \multicolumn{10}{|l|}{Tarefas em andamento}&\cellcolor{yellow} \\
\hline \multicolumn{10}{|l|}{Tarefas não realizadas}&\cellcolor{red} \\
\hline
\end{tabular}
\caption{Cronograma do Trabalho}
\label{table:cronograma}
\end{table}