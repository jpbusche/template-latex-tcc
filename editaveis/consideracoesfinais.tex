\chapter[Considerações Finais]{Considerações Finais}
Neste capítulo são discutidos as conclusões levantadas sobre o desenvolvimento do projeto.

Sobre a implementação da arquitetura proposta, é concluido que, apesar dos gargalos encontrados do decorrer do projeto, a arquitetura proposta, mostrou-se capaz de completar as funcionalidades a ela atribuída. Após a divisão dos tipos de dados, em estáticos e temporais, o uso do Elasticsearch tornou-se muito mais vantajoso, o que aumento a \textit{performance} de inserção dos dados, como também diminuiu o número de gargalos do projeto. Finalmente, posso concluir que a proposta de arquitetura de extração de dados do mercado de jogos precisa de algumas modificações, porém, consegue criar um lugar onde todos esses dados se concentre.

\section{Trabalhos Futuros}
Nesta seção são levantados possíveis trabalhos futuros no desenvolvimento do projeto.
\begin{itemize}
	\item Inserção de novas fontes de dados, disponibilizando uma maior número de informações sobre os jogos.
	\item Implementação de uma API RESTful, possibilitando uma maneira mais fácil de obtenção destes dados.
	\item Criação de métricas de negócios a partir dos dados extraídos.
	\item Desenvolvimento de uma plataforma web para disponibilização destas métricas.
\end{itemize}