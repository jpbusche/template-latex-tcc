\chapter[Considerações Finais]{Considerações Finais}
Neste capítulo são discutidos as conclusões levantadas sobre o desenvolvimento do projeto.

Durante a implementação da arquitetura, diversos desafios apareceram, muitos desses em consequência da pouco conhecimento da linguagem de programação escolhida. Dentre esses desafios, houve um que não foi resolvido no tempo dedicado a esse trabalho, sendo ele, a implementação da funcionalidade de manter os registros dos dados temporais.

Com mais maturidade sobre as ferramentas utilizadas no decorrer do projeto, percebo, que havia maneiras mais eficientes de desenvolver a arquitetura, e, por consequência, maneiras de implementar o registro de dados temporias, como, a cada chamada das tarefas de atualização das informações, fosse criado um novo documento no Elasticsearch, não inserido num já existente.

Finalmente, conclui-se que a proposta de arquitetura de extração de dados do mercado de jogos, insere e disponibilizá esses dados num só lugar, além de possibilitar a criação de ferramentas para a utilização dos mesmos.

\section{Trabalhos Futuros}
Nesta seção são levantados possíveis trabalhos que podem ser desenvolvidas a partir do presente. São eles:
\begin{itemize}
	\item inserção de novas fontes de dados, disponibilizando uma maior número de informações sobre os jogos;
	\item implementação de uma API RESTful, possibilitando uma maneira mais fácil de obtenção destes dados;
	\item criação de métricas de negócios a partir dos dados extraídos;
	\item desenvolvimento de uma plataforma web para disponibilização destas métricas.
\end{itemize}