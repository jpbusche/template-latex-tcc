\chapter*[Introdução]{Introdução}
\addcontentsline{toc}{chapter}{Introdução}
A indústria de jogos é uma das áreas de mercado mais lucrativa atualmente. No mercado brasileiro isto não seria diferente: sendo apenas o 13º maior mercado, no ano de 2017 movimentou 1,3 bilhões de dólares de acordo com o Newzoo \cite{newzoo_brasil}, empresa que estuda o mercado de jogos, um valor pequeno se comparado ao mercado chinês, o maior mercado de jogos do mundo, o qual movimentou 27,5 bilhões de dólares no ano de 2017 \cite{newzoo_china}.

Sendo um mercado bastante lucrativo, houve um grande aumento no número de empresas desenvolvedoras de jogos. No mercado brasileiro, houve um aumento de 600 \% entre 2008 e 2016 no número de desenvolvedoras \cite{desenvolvedoras_crescimento}. Este aumento não se limita apenas ao mercado brasileiro: de acordo com o site Steamspy \cite{steam_spy}, plataforma web que exibi dados de jogos, apenas no ano de 2017 foram lançados 7.672 jogos na plataforma Steam, plataforma onde desenvolvedoras disponibilizam seus jogos, uma média de 21 jogos por dia. Porém com este grande crescimento fica cada vez mais difícil desenvolver um jogo que seja aceito pela comunidade de jogadores e que venda um número considerável de cópias.

Levando em conta a dificuldade de se criar um jogo nos tempos atuais, muitas desenvolvedoras iniciantes buscam por métricas que auxiliem na hora do desenvolvimento. Porém essas métricas não estão de fácil acesso, as exigindo assim, que as equipes montem essas métricas com os dados que possuem ou apenas as ignorem e desenvolvam o jogo sem elas. Esta dificuldade de achar métricas ocorre principalmente pela falta de uma ferramenta especializada nesta área.

Por isso o objetivo deste trabalho e a elaboração de métricas de negócios, como o número médio de vendas de determinado gênero, e a criação de uma plataforma web onde essas métricas serão disponibilizadas em conjunto com sugestões para as desenvolvedoras.

\section*{Objetivos}
	O objetivo deste trabalho é desenvolver uma plataforma que disponibilize métricas e sugestões para desenvolvedoras de jogos.
	
	Os objetivos espécificos são:
	\begin{itemize}
		\item elaborar métricas a partir das informações dos jogos;
		\item automatizar as sugestões de melhorias para a concepção do jogo.;
		\item elaborar uma arquitetura responsável por fazer a extração e manipulação dos dados dos jogos;
		\item desenvolver um software responsável por extrair e manipular os dados que serão utilizados.
	\end{itemize}
\section*{Estrutura do Documento}
Este documento será dividido em 4 capítulos. O Capítulo 1 é responsável pelo referencial teórico do projeto, com um estudo sobre \textit{game analytic}s e \textit{business intelligence}, e análise de concorrentes e ferramentas que serão utilizadas no projeto. O Capítulo 2 é responsável pela metodologia, mostrando como será feito o projeto. O Capítulo 3 é responsável pelos resultados obtidos no projeto. O Capítulo 4 é responsável pela conclusão do projeto, também será mostrados possíveis trabalhos futuros.