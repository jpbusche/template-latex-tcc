\chapter[Introdução]{Introdução}
A indústria de jogos é uma das áreas de mercado mais lucrativa atualmente. No mercado brasileiro isto não seria diferente: sendo apenas o 13º maior mercado, no ano de 2017 movimentou 1,3 bilhões de dólares de acordo com o Newzoo \cite{newzoo_brasil}, empresa que estuda o mercado de jogos. Um valor pequeno se comparado ao mercado chinês, o maior mercado de jogos do mundo, o qual movimentou 27,5 bilhões de dólares no ano de 2017 \cite{newzoo_china}.

No mercado brasileiro, houve um aumento de 600 \% entre 2008 e 2016 no número de desenvolvedoras \cite{desenvolvedoras_crescimento}. Este aumento não se limita apenas ao mercado brasileiro, de acordo com o site Steamspy \cite{steam_spy}, plataforma web que exibe dados de jogos, apenas no ano de 2017 foram lançados 7.672 jogos na plataforma Steam\footnote[1]{\url{https://store.steampowered.com/}}, uma plataforma que disponibiliza jogos para downloads, uma média de 21 jogos por dia. Porém, à medida que o mercado desenvolvedor cresce, aumenta-se a concorrência no meio, tornando cada vez mais difícil a aceitação do profissional pela comunidade em razão da variedade de opções  na hora da compra fazendo com que algumas \textit{features} se tornem diferencias, como por exemplo \textit{multiplayer}, localização, dentre outras.

Levando em conta a dificuldade de se criar um jogo nos tempos atuais, muitas desenvolvedoras \textit{indies} buscam por métricas que auxiliem na hora do desenvolvimento. Estas métricas, apesar de estarem disponíveis no mercado, necessitam de um grande número de recursos humanos e tempo para extrair informações que agreguem valor ao desenvolvimento. Essas desenvolvedoras, por falta deste tipo de recursos, muitas vezes pagam para outras empresas fazerem essas análises ou desenvolvem jogos sem as análises.

Desta forma o objetivo deste trabalho é a criação de uma arquitetura que irá extrair dados sobre o mercado de jogos, disponibiliza-los num só lugar, possibilitando a criação de ferramentas de análise dos dados.

\section{Objetivos}
	O objetivo deste trabalho é uma arquitetura de extração de variadas fontes para a criação de um banco comum.
	
	Os objetivos espécificos são:
	\begin{itemize}
		\item Elaborar uma arquitetura responsável por fazer a extração, manipulação dos dados dos jogos;
		\item Elaborar uma rotina para a extração constantes dos dados;
		\item Desenvolver um software responsável por extrair, manipular e inserir os dados dos jogos num banco de dados.
	\end{itemize}
\section{Estrutura do Documento}
Este documento será dividido em 5 capítulos, sendo o primeiro a introdução. O Capítulo 2 é responsável pelo referencial teórico do projeto e análise de concorrentes. O Capítulo 3 é responsável pela metodologia, mostrando como será feito o projeto. O Capítulo 4 é responsável pelos resultados obtidos no projeto. O Capítulo 5 é responsável pela conclusão do projeto, onde também será mostrados possíveis trabalhos futuros.