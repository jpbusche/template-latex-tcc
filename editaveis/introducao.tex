\chapter[Introdução]{Introdução}
A indústria de jogos é uma das áreas de mercado mais lucrativa atualmente. No mercado brasileiro isto não seria diferente: sendo apenas o 13º maior mercado, no ano de 2017 movimentou 1,3 bilhões de dólares de acordo com o Newzoo \cite{newzoo_brasil}, empresa que estuda o mercado de jogos. Um valor pequeno se comparado ao mercado chinês, o maior mercado de jogos do mundo, o qual movimentou 27,5 bilhões de dólares no ano de 2017 \cite{newzoo_china}.

Sendo um mercado bastante lucrativo, houve um grande aumento no número de empresas desenvolvedoras de jogos. No mercado brasileiro, houve um aumento de 600 \% entre 2008 e 2016 no número de desenvolvedoras \cite{desenvolvedoras_crescimento}. Este aumento não se limita apenas ao mercado brasileiro: de acordo com o site Steamspy \cite{steam_spy}, plataforma web que exibi dados de jogos, apenas no ano de 2017 foram lançados 7.672 jogos na plataforma Steam, plataforma onde desenvolvedoras disponibilizam seus jogos, uma média de 21 jogos por dia. Porém com este grande crescimento de concorrência fica cada vez mais difícil desenvolver um jogo que seja aceito pela comunidades de jogadores, assim, os jogadores tem muitas opções na hora comprar, o que faz que com algumas \textit{features} se tornem diferencias, como multyplayer, localização, entre outros.

Levando em conta a dificuldade de se criar um jogo nos tempos atuais, muitas desenvolvedoras \textit{indies} buscam por métricas que auxiliem na hora do desenvolvimento. Estas métricas, apesar de estarem disponíveis no mercado, necessitam um grande numero de recursos humanos/tempo para extrair informações que agreguem valor ao desenvolvimento. Desenvolvedoras \textit{indies}, geralmente por falta deste tipo de recursos, muitas vezes pagam para outras empresas fazerem essas análises ou desenvolvem jogos sem as análises. 

Por isso o objetivo deste trabalho e a criação de uma plataforma web onde será disponibilizada, de uma forma mais simples, diferentes métricas e agregações de dados, que permite o usuário moldar a necessidade dele. Incluindo \textit{insights} sobre as métricas.

\section{Objetivos}
	O objetivo deste trabalho è desenvolver um banco de dados inicial, com os dados de diversas fontes, para que futuramente possa ser criado as métricas de negócio.
	
	Os objetivos espécificos são:
	\begin{itemize}
		\item elaborar uma arquitetura responsável por fazer a extração, manipulação dos dados dos jogos;
		\item elaborar uma rotina para a extração constantes dos dados;
		\item desenvolver um software responsável por extrair, manipular e inserir os dados dos jogos num banco de dados.
	\end{itemize}
\section{Estrutura do Documento}
Este documento será dividido em 5 capítulos, sendo o primeiro a introdução. O Capítulo 2 é responsável pelo referencial teórico do projeto e análise de concorrentes. O Capítulo 3 é responsável pela metodologia, mostrando como será feito o projeto. O Capítulo 4 é responsável pelos resultados obtidos no projeto. O Capítulo 5 é responsável pela conclusão do projeto, onde também será mostrados possíveis trabalhos futuros.