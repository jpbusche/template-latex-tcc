\chapter*[Introdução]{Introdução}
\addcontentsline{toc}{chapter}{Introdução}
A indústria de jogos é uma das áreas de mercado mais lucrativa atualmente. No mercado brasileiro isto não seria diferente, sendo apenas o 13º maior mercado, no ano de 2017 movimentou 1,3 bilhões de dólares de acordo com o site de Newzoo[1], e isto não é nada se for comparado ao mercado chinês, que é o maior mercado de jogos do mundo, onde no ano de 2017 movimentou 27,5 bilhões de dólares[2].

Sendo um mercado bastante lucrativo, houve um grande aumento no número de empresas desenvolvedoras de jogos, se formos foca apenas no mercado brasileiro, houve um aumento de 600\% entre 2008 e 2016 no número de desenvolvedoras[3]. Este aumento não se limita apenas ao mercado brasileiro, de acordo com o site Steamspy[4], apenas no ano de 2017 foram lançados 7672 jogos na plataforma Steam, isso dá em média 21 jogos por dia. Porém com este grande crescimento fica cada vez mais difícil desenvolver um jogo que seja bem aceito pela comunidade de jogadores e que venda um número considerável de cópias.

Levando em conta a dificuldade de se criar um jogo nos tempos atuais, muitas desenvolvedoras iniciantes buscam por métricas que auxiliem na hora do desenvolvimento, porém muitas vezes essas métricas não estão de fácil acesso, assim exigindo que as equipes monte essas métricas com os dados que possui ou apenas ignorem e desenvolvam o jogo do mesmo jeito. Está dificuldade de achar as métricas ocorre principalmente pela falta de uma ferramenta especializada nesta área.

Por isso o objetivo deste projeto e a elaboração de métricas de negócios, como o número médio de vendas de determinado gênero, e a criação de uma plataforma web onde essas métricas serão disponibilizadas em conjunto com pequenas sugestões para as desenvolvedoras.

\section*{Objetivos}
	\subsection*{Objetivo Geral}
	O objetivo deste trabalho é desenvolver uma plataforma que disponibilize métricas e sugestões para o auxílio de desenvolvedores de jogos
	\subsection*{Objetivos Específicos}
	\begin{itemize}
		\item Elaborar métricas a partir das informações dos jogos.
		\item Automatizar as sugestões de melhorias para a concepção do jogo.
		\item Elaborar uma arquitetura responsável por fazer a extração, manipulação dos dados dos jogos.
		\item Desenvolver um software responsável por extrair e manipular os dados dos logs.
	\end{itemize}
\section*{Estrutura do Documento}
Este documento será dividido em 7 capítulos. O primeiro capítulo é responsável pelo referencial teórico do projeto, onde terá um estudo sobre game analytics e business intelligence, e análise de concorrentes e ferramentas que serão utilizadas no projeto. O segundo capítulo é responsável pela metodologia, neste capítulo será mostrado como será feito o projeto. O terceiro capítulo é responsável pelos resultados obtidos no projeto. O quarto capítulo é responsável pela conclusão do projeto, também será mostrados possíveis trabalhos futuros. O três capítulos restantes terão as referências, apêndices e anexos existentes do projeto.