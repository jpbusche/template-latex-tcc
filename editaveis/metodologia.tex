\chapter*[Metodologia]{Metodologia}
A metodologia escolhida para o desenvolvimento do projeto será a metodologia ágil, mais especificamente será utilizado o Kanban. A escolha dessa metodologia se dá pelas seguintes características: o projeto será desenvolvido de maneira incremental, ou seja, ele poderá ser modificado no decorrer da implementação; a equipe consiste em apenas uma pessoa, o que descarta a possibilidade de utilizar o Scrum ou XP; o escopo do projeto será divididos em tarefas, e a utilização do Kanban facilita no descobrimento de gargalos.

A metodologia Kanban surgiu no Japão com o TPS[1] para controlar a fabricação de automóveis e foi inserida no meio de desenvolvimento de software no ano de 2007. Kanban é um termo japonês para sinal visual, uma das grandes características dessa metodologia é evidenciar os problemas existentes no processo. 

A metodologia ágil surgiu no ano de 2001, com a reunião de especialistas em processos de desenvolvimento de software para discutir maneiras de melhorar o desempenho em projetos, com isso foi criado o Manifesto Ágil. Uma das características das metodologias ágeis são sua capacidade de adaptar a novos fatores durante o desenvolvimento do projeto, ao invés de tentar prever o que pode acontecer e o que não pode
\section*{Requisitos}
Os requisitos do projeto serão classificados em três níveis de hierarquia:  épicos, features e user stories. Os épicos são compostos por features e representam macro entregáveis do projeto, geralmente são descritos como o nível mais abstrato. As features são agrupamentos de user stories e representam funcionalidades do sistema. User stories são pequenas partes de uma funcionalidade do sistema e que serão implementadas, user stories também podem ser classificadas como technical stories, estas technical stories representam atividades que não agregam muito valor ao cliente, porém elas agregam numa melhora interna no software.

No contexto do projeto as features serão equivalentes às milestones, e as stories (users ou technical) serão as issues. Para uma melhor visualização dos requisitos do projeto, será montado uma matriz de rastreabilidade que é mostrada abaixo:
