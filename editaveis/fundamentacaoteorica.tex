\chapter[Fundamentação Teórica]{Fundamentação Teórica}
Neste capítulo será abordada a fundamentação teórica para o entendimento do propósito da implementação do projeto. Nele são explicados os conceitos de \textit{application programming interface}, \textit{business intelligence} e \textit{game analytics}, bem como serão expostos softwares similares ao desenvolvido.

\section{\textit{Application Programming Interface}}
A sigla API (\textit{Application Programming Interface}) é um conjunto de rotinas e padrões de programação para acesso a um aplicativo de software ou plataforma baseado na Web. A utilização destas rotinas servem principalmente para que softwares externos tenham acesso a funcionalidade de um produto, sem que esse tenha que envolver-se em detalhes da implementação \cite{api}.

Para a utilização de APIs Webs existem várias arquiteturas que podem ser utilizadas, a mais comum é a arquitetura REST (\textit{Representational State Transfer}), que define um conjunto de restrições e propriedades baseadas no HTTP (\textit{Hypertext Transfer Protocol}).
\subsection{\textit{Representational State Transfer}}
A sigla REST, originalmente, se referia a um conjunto de princípios de arquitetura, nos tempos atuais é utilizada no sentido mais amplo, descrevendo qualquer interface web que use XML (\textit{Extensible Markup Language}) e HTTP, sem as abstrações adicionais dos protocolos baseados em padrões de trocas de mensagem. De acordo com Fielding a própria \textit{World Wide Web}, utilizou REST como base para o desenvolvimento do protocolo HTTP, sendo assim possível projetar um sistema de serviços web com a arquitetura REST \cite{fielding}.

A arquitetura REST possui quatro princípios, sendo eles:
\begin{itemize}
	\item \textbf{Protocolo cliente/servidor sem estado}: cada mensagem HTTP contém toda a informação necessária para compreender o pedido.
	\item \textbf{Operações bem definidas}: que se aplicam a todos os recursos de informações, as mais importantes são \textit{\textbf{POST}}, \textit{\textbf{GET}}, \textit{\textbf{PUT}} e \textit{\textbf{DELETE}}.
	\item \textbf{Sintaxe universal}: para identificar os recursos, na arquitetura REST este recurso é direcionado pela sua URI(\textit{Uniform Resource Identifier}).
	\item \textbf{Hipermídia}: para a informação da aplicação, como para as transições de estado da aplicação, normalmente são representados pelo HTTP e XML
\end{itemize}

\section{\textit{Business Intelligence}}
BI (\textit{Business Intelligence}) é uma técnica de gerenciamento de dados, possibilitando a extração de valores de negócios das grandes pilhas de dados disponibilizadas, sejam elas de estruturas de dados ou de arquivos \cite{business_intelligence}. As análises deste dados proporcionam a criação de conhecimento, dessa maneira, originou-se vantagens inimagináveis no mercado de negócios. 

Uma das maiores vantagens da utilização de BI está na otimização do processo de tomada de decisão, pois num mundo ideal, cada escolha deve ser a mais otimizada, ou seja, a que tem a melhor \textit{performance}. Outras vantagens vão, desde a criação de póliticas de escolhas para diferentes cenários de negócios, à exposição e exploração de conhecimentos \cite{business_intelligence}. 

A utilização de BI no âmbito de trabalho apresenta uma evolução na \textit{performance} nas seguintes dimensões de negócio: valor financeiro, valor de produtividade, valor de confiança e valor de risco.

O valor financeiro é associado ao crescimento da lucratividade, sejam elas derivadas de custos ou do aumento de receitas.

O valor de produtividade é associado a diminuição da carga de trabalho, diminuição do tempo necessário para a execução de processos ponta-a-ponta e no aumento da porcentagem de produtos de alta qualidade.

O valor de confiança, como maior satisfação do cliente, funcionário ou fornecedor, assim como aumento na confiança de previsões, consistência operacional e relatórios gerenciais, reduções no tempo gasto com “paralisia de análise” e melhor resultados de decisões.

O valor de risco é associado com a uma melhor visibilidade da exposição do crédito, confiança no investimentos em capitais e conformidade auditável com a jurisdição e normas e regulamentos da indústria.

The Data Warehousing Institute \cite{tdwi}, uma instituição especializada na educação e treinamento em armazenamento de dados, define que BI são os processos, as tecnologias e as ferramentas necessárias para transformar dados em informação, informação em conhecimento e conhecimento em planos que dirigem rentáveis planos de negócios, ou seja BI engloba armazenamento de dados, ferramentas de análiticas e gestão de informação/conhecimento. 

BI geralmente considera as informações dos dados como ativos, por isso, pode-se valer a pena examinar o uso de informações no contexto de como o valor é criado dentro de uma organização. Para tanto existem três tipos diferentes de perspectivas.

Perspectiva funcional, nela os processos focam nas tarefas relacionadas a algum tipo particular de negócio, como vendas, \textit{marketing}, dentre outras. Processos funcionais confiam nos dados que operam dentro dos padrões de atividades comerciais.

Perspectiva interfuncional funciona como um aglomerado de processos funcionais, refletindo em informações mais complexas. Para esta perspectiva a atividade é um sucesso quando todas as tarefas são completadas. Pela sua própria natureza, os processos envolvidos compartilham informações em diferentes funções, e o sucesso é medido tanto em termos de conclusão bem-sucedida, bem como as características do desempenho geral.

Perspectiva empresarial funciona num ponto de vista organizacional, observando as características de desempenho dos processos interfuncionais, pode-se informar arquitetos empresariais e analistas de negócios as maneiras nas quais a organização pode mudar e melhorar o jeito com que as coisas são feitas. Neste ponto de vista, o dado não é mais usado apenas para executar negócios, mas para melhorá-los.

\section{\textit{Game Analytics}}
O desenvolvimento de jogos hoje pode se mostrar como um desafio gigantesco, e parte deste desafio se dá em razão do grande número de jogos publicados. Para auxiliar as desenvolvedoras a criarem jogos de maneira eficiente foram criados várias ferramentas e técnicas, como o \textit{analytics}.

\textit{Analytics} é o processo de descobrir e comunicar padrões em dados, solucionando problemas de negócios ou suportando decisões de gerenciamento de empresas. Esta metodologia possui seus fundamentos em mineração de dados, na matemática, estatística, programação e operações de busca, como também na visualização dos dados, de forma a comunicar padrões relevantes. Vale mencionar que o \textit{analytics} não é apenas perguntar e relatar dados de BI, e sim análises atuais daqueles dados \cite{analytics}. 

\textit{Game analytics} é uma aplicação do \textit{analytics} para o contexto de desenvolvimento de jogos \cite{game_analytics}. Um dos maiores benefícios em utilizar o \textit{game analytics} é o suporte na hora de fazer decisões em todos os níveis e áreas organizacionais. Este método é direcionado tanto na análise de um jogo com um produto, quanto na análise de um jogo como projeto.

A aplicação padrão do \textit{game analytics} é na hora de informar o GUR (\textit{Game User Research}). GUR é a aplicação de várias técnicas e metodologias para avaliar a maneira na qual os usuários jogam, e o nível de interação entre o jogador e o jogo. Vale mencionar que \textit{game analytics} não é só GUR, já que este é focado nos dados obtidos a partir dos usuários, e o \textit{game analytics} considera todos os tipo de dados obtidos no desenvolvimento do jogo.
\subsection{Telemetria}
Telemetria são os dados obtidos à distância, geralmente digitais. No contexto de jogos, telemetria seria a transmissão de dados por parte do produto acerca da interação com o usuário.

Telemetria de jogos é o uso da telemetria para obtençao de dados durante o desenvolvimento ou evolução de um jogo, e isto inclui o monitoramento e análise: de servidores, dispositivos celulares e comportamento dos usuários. A fonte que produz mais dados por telemetrias, são os de usuário, por exemplo, interação com jogos, comportamento de compra e interações com outros jogadores ou aplicativos \cite{telemetry}.
\subsection{\textit{Game Metrics}}
Em sua forma pura, os dados obtidos a partir da telemetria, não são de muito auxílio, por isso estes dados devem ser transformado em várias métricas interpretativas, como: o número de jogadores ativos por dia, \textit{bugs} arrumados por semana, entre outras. Essas métricas são chamadas de \textit{game metrics}. \textit{Game metrics} possuem os mesmo potencias que outras fontes de BI. \textit{Game metrics} geralmente são definidas como um medição quantitativa de um ou mais atributos, ou objetos que operem no contexto de um jogo.

Métricas podem ser variáveis ou  agregações mais complexas, como a soma de várias variáveis, em outras palavras, as métricas podem ser simples variáveis que geram uma análise básica, ou a combinação de várias variáveis para gerar uma análise mais complexa e completa. Métricas que não estão relacionadas diretamente ao jogo, são chamadas de métricas de negócios. Durante a utilização do \textit{game analytics} é essencial a distinção entre as métricas de negócio e as \textit{game metrics}.

As game metrics foram categorizadas em três tipos de acordo com Mellon \cite{game_metrics}, sendo elas as de usuários, de \textit{performance} e de processo.
Métricas de usuário são métricas relacionadas aos usuários que utilizam aquele determinado jogo, pela perspectiva de jogadores, ou de clientes. A perspectiva de cliente é utilizada quando as métricas são relacionadas a receita. A perspectiva de jogador é utilizada para investigar como é a interação das pessoas com o sistema do jogo e seus componentes.

Métricas de \textit{performance} são métricas relacionadas a \textit{performance} da tecnologia e arquitetura utilizada no jogo, muito relevantes para jogos onlines. Essas métricas geralmente são utilizadas no monitoramento dos impactos causados por alguma atualização no jogo.

Métricas de processo são métricas relacionadas ao processo de desenvolvimento de jogos. Similares as métricas de \textit{performance}, são utilizadas para gerenciar e monitorar métodos que foram adotados, ou que foram utilizados na hora do desenvolvimento do jogo. 
\section{Softwares Correlatos}
Nesta seção são levantado os softwares correlatos, que possuem características ou objetivos parecidos com a plataforma a ser desenvolvida. Os principais são o Steam DB, uma ferramenta que disponibiliza informações sobre o banco de dados da Steam; a Steam Spy, uma plataforma web que disponibiliza informações sobre os jogos e suas vendas por região e o DFC Intelligence, uma ferramenta de pesquisa sobre o mercado de jogos.
\subsection{Steam DB}
Steam DB\footnote[1]{\url{https://steamdb.info/}} é uma ferramenta \textit{open source} com objetivo de dar um melhor conhecimento sobre os jogos e suas atualizações disponíveis no banco de dados da Steam \cite{steam_db}.

Steam DB disponibiliza seus dados de maneira gratuita, porém para ter acesso a esses dados é preciso fazer login com uma conta da Steam. Estes dados estão dispostos de diversas maneiras, como \textit{rankings} de diferentes categorias, gráficos de um jogo específico e até informações sobre o usuário. Porém, não é possível fazer contribuições a ferramenta, impossibilitando a inserção de novos dados e de novas métricas.
\subsection{Steam Spy}
Steam Spy\footnote[2]{\url{http://steamspy.com/}} é uma plataforma Web que a partir de informações sobre os usuários da Steam, disponibiliza informações como o número de donos de determinado jogo ou vendas por região \cite{steam_spy}. Um dos objetivos especificados pelo Steam Spy é o auxílio a desenvolvedores \textit{indies}.

Steam Spy dispõe seus dados de duas maneiras, uma delas é pela plataforma, como gráficos e \textit{rankings} por categoria. A outra forma e por meio de sua API, disponibilizada para desenvolvedores criarem suas próprias ferramentas. Porém, para ter acesso total ao seus gráficos é preciso pagar por eles.

\subsection{DFC Intelligence}
DFC Intelligence\footnote[3]{\url{https://www.dfcint.com/}} é uma ferramenta que auxiliam desenvolvedoras a tomar conhecimentos sobre estatísticas de seus jogos no mercado.

Para as desenvolvedoras que contratem o serviço da DFC Intelligence receberão métricas sobre seus jogos, e também sobre informações sobre como o mercado está agindo para jogos semelhantes. Porém, para que uma desenvolvedora possa usufruir de seus dados é preciso pagar uma taxa de consulta, além do pagamento pela ferramenta.