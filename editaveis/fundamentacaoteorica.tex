\chapter[Fundamentação Teórica]{Fundamentação Teórica}
Neste capítulo será abordada a fundamentação teórica para o entendimento do propósito da implementação do projeto. Nele são explicados os conceitos de \textit{application programming interface}, \textit{business intelligence} e \textit{game analytics}, bem como serão expostos softwares similares ao desenvolvido.

\section{\textit{Application Programming Interface}}
A sigla API (\textit{Application Programming Interface}) é um conjunto de rotinas e padrões de programação para acesso a um aplicativo de software ou plataforma baseado na Web. A utilização destas rotinas servem principalmente para que softwares externos tenham acesso a funcionalidade de um produto, sem que esse tenha que envolver-se em detalhes da implementação \cite{api}.

Para a utilização de APIs Webs existem várias arquiteturas que podem ser utilizadas, a mais comum é a arquitetura REST (\textit{Representational State Transfer}), que define um conjunto de restrições e propriedades baseadas no HTTP (\textit{Hypertext Transfer Protocol}).
\subsection{\textit{Representational State Transfer}}
A sigla REST, originalmente, se referia a um conjunto de princípios de arquitetura, nos tempos atuais é utilizada no sentido mais amplo, descrevendo qualquer interface web que use XML (\textit{Extensible Markup Language}) e HTTP, sem as abstrações adicionais dos protocolos baseados em padrões de trocas de mensagem. De acordo com Fielding \cite{fielding} a própria \textit{World Wide Web}, utilizou REST como base para o desenvolvimento do protocolo HTTP, sendo assim possível projetar um sistema de serviços web com a arquitetura REST.

A arquitetura REST possui quatro princípios principais, sendo eles:
\begin{itemize}
	\item \textbf{Protocolo cliente/servidor sem estado}: cada mensagem HTTP contém toda a informação necessária para compreender o pedido.
	\item \textbf{Operações bem definidas}: que se aplicam a todos os recursos de informações, as mais importantes são \textit{\textbf{POST}}, \textit{\textbf{GET}}, \textit{\textbf{PUT}} e \textit{\textbf{DELETE}}.
	\item \textbf{Sintaxe universal}: para identificar os recursos, na arquitetura REST este recurso é direcionado pela sua URI(\textit{Uniform Resource Identifier}).
	\item \textbf{Hipermídia}: para a informação da aplicação, como para as transições de estado da aplicação, normalmente são representados pelo HTTP e XML
\end{itemize}

\section{\textit{Business Intelligence}}
No decorrer dos anos houve uma grande mudança em relação a criação, coleta e uso dos dados. Enquanto teve uma grande evolução na maneira que esses dados eram gerenciados, sempre existiu um desejo de extrair valores de negócios nas grandes pilhas de dados, que são capturadas hoje em diversas fontes, sejam elas estruturas de dados ou arquivos.

Os resultados da análise destas pilhas de dados propiciaram a formação de conhecimento a respeito, dessa maneira, criou-se uma vantagem no mercado de negócios inimagináveis. Com esse conhecimento foi possível criar políticas de escolhas de ações para negócios em diferentes cenários. A exposição e exploração destes conhecimento é apenas uma das vantagens do uso de BI (\textit{Business Intelligence}).

Uma das maiores vantagens do uso do BI está na otimização do processo de tomada de decisão, pois para cada processo de negócios é associado a sua \textit{performance}, e ,num mundo perfeito, cada escolha deve ser a mais otimizada, ou seja, a que tem a melhor \textit{performance} \cite{business_intelligence}.

A utilização de BI no âmbito de trabalho apresenta uma evolução na \textit{performance} nas seguintes dimensões de negócio:
\begin{itemize}
		\item No valor financeiro associado ao crescimento da lucratividade, sejam elas derivadas de custos ou do aumento de receitas.
		\item No valor de produtividade associado a diminuição da carga de trabalho, diminuição do tempo necessário para a execução de processos ponta-a-ponta e no aumento da porcentagem de produtos de alta qualidade.
		\item No valor de confiança, como maior satisfação do cliente, funcionário ou fornecedor, assim como aumento na confiança de previsões, consistência operacional e relatórios gerenciais, reduções no tempo gasto com “paralisia de análise” e melhor resultados de decisões.
		\item No valor de risco associado com a uma melhor visibilidade da exposição do crédito, confiança no investimentos em capitais e conformidade auditável com a jurisdição e normas e regulamentos da indústria.
\end{itemize}
The Data Warehousing Institute \cite{tdwi}, uma instituição especializada na educação e treinamento em armazenamento de dados, define que BI são os processos, as tecnologias e as ferramentas necessárias para transformar dados em informação, informação em conhecimento e conhecimento em planos que dirigem rentáveis planos de negócios, ou seja BI engloba armazenamento de dados, ferramentas de análiticas e gestao de informação/conhecimento. 

\textit{Business intelligence} geralmente considera as informações dos dados como ativos, por isso, pode-se valer a pena examinar o uso de informações no contexto de como o valor é criado dentro de uma organização. Para isso existem três tipos diferentes de perspectivas, sendo elas:
\begin{itemize}
		\item \textbf{Perspectiva funcional}: Neste tipo de perspectiva, os processos focam nas tarefas relacionadas a algum tipo particular de negócio, como vendas, \textit{marketing}, entre outros. Processos funcionais confiam nos dados que operam dentro dos padrões de atividades comerciais.
		\item \textbf{Perspectiva interfuncional}: Como a maioria das empresas funcionam como um aglomerado de processos funcionais, e isto reflete em informações mais complexas. Para esta perspectiva a atividade é um sucesso quando todas as tarefas são completadas. Pela sua própria natureza, os processos envolvidos compartilham informações em diferentes funções, e o sucesso é medido tanto em termos de conclusão bem-sucedida, bem como as características do desempenho geral.
		\item \textbf{Perspectiva empresarial}: Num ponto de vista organizacional e observando as características de desempenho dos processos interfuncionais, pode-se informar arquitetos empresariais e analistas de negócios as maneiras na qual a organização pode mudar e melhorar o jeito que as coisas são feitas. Neste ponto de vista, o dado não é mais usado apenas para executar negócios, mas para melhorá-los.
\end{itemize}

\section{\textit{Game Analytics}}
O desenvolvimento de jogos hoje pode se mostrar como um desafio gigantesco, e parte deste desafio se dá pelo fato do grande número de jogos publicados. Para auxiliar as desenvolvedoras a criarem jogos de maneira eficiente foram criados várias ferramentas e técnicas, como o \textit{analytics}.

\textit{Analytics} é o processo de descobrir e comunicar padrões em dados, solucionando problemas de negócios ou suportar decisões de gerenciamento de empresas. Esta metodologia possui seus fundamentos em mineração de dados, na matemática, estatística, programação e operações de busca, como também na visualização dos dados, de forma a comunicar padrões relevantes. Vale mencionar que o \textit{analytics} não é apenas perguntar e relatar dados de BI, e sim análises atual daqueles dados \cite{analytics}. 

\textit{Game analytics} é uma aplicação do \textit{analytics} para o contexto de desenvolvimento de jogos \cite{game_analytics}. Um dos maiores benefícios em utilizar o \textit{game analytics} é o suporte na hora de fazer decisões em todos os níveis e áreas organizacionais. Este método é direcionado tanto como a análise de um jogo com um produto, tanto como a análise de um jogo como projeto.

A aplicação padrão do \textit{game analytics} é na hora de informar o GUR (\textit{Game User Research}). GUR é a aplicação de várias técnicas e metodologias para avaliar a maneira na qual os usuários jogam, e o nível de interação entre o jogador e o jogo. Vale mencionar que \textit{game analytics} não é só GUR, já que o este é focado nos dados obtidos a partir dos usuários, e o \textit{game analytics} considera todos os tipo de dados obtidos no desenvolvimento do jogo.
\subsection{Telemetria}
Telemetria são os dados obtidos à distância, geralmente digitais, porém qualquer dado transmitido à distância é telemetria. No contexto de jogos, telemetria seria algum jogo transmitindo dados sobre a interação do usuário com o jogo.

Telemetria de jogos é o termo utilizado para qualquer dado obtido a distância que pertence durante o desenvolvimento ou evolução de um jogo, e isto inclui o monitoramento e análise: de servidores, dispositivos celulares e comportamento dos usuários. A fonte que produz mais dados por telemetrias, são os de usuário, por exemplo, interação com jogos, comportamento de compra e interações com outros jogadores ou aplicativos \cite{telemetry}.
\subsection{\textit{Game Metrics}}
Em sua forma pura, os dados obtidos a partir da telemetria, não são de muito auxílio, por isso estes dados devem ser transformado em várias métricas interpretativas, como: o número de jogadores ativos por dia, bugs arrumados por semana, entre outros. Essas métricas são chamadas de \textit{game metrics}. \textit{Game metrics} possuem os mesmo potencias que outras fontes de BI. \textit{Game metrics} geralmente são definidas como um medição quantitativa de um ou mais atributos, ou objetos que operem no contexto de um jogo.

Métricas podem ser variáveis ou  agregações mais complexas, como a soma de várias variáveis, em outras palavras, as métricas podem ser simples variáveis que geram uma análise básica, ou a combinação de várias variáveis para gerar uma análise mais complexa e completa. Métricas que não estão relacionadas diretamente ao jogo, são chamadas de métricas de negócios. Durante a utilização do \textit{game analytics} é essencial a distinção entre as métricas de negócio e as \textit{game metrics}.

As game metrics foram categorizadas em três tipos por Mellon \cite{game_metrics}, ou seja, as \textit{game metrics} podem ser definidas como:
\begin{itemize}
	\item \textbf{Métricas de usuário}: São métricas relacionadas aos usuários que utilizam aquele determinado jogo, pela perspectiva de jogadores, ou de clientes. A perspectiva de cliente é utilizada quando as métricas são relacionadas a receita. A perspectiva de jogador é utilizada para investigar como é a interação das pessoas com o sistema do jogo e seus componentes.
	\item \textbf{Métricas de performance}: São métricas relacionadas a performance da tecnologia e arquitetura utilizada no jogo, muito relevantes para jogos onlines. Essas métricas geralmente são utilizadas no monitoramento dos impactos causados por alguma atualização no jogo.
	\item \textbf{Métricas de processo}: São métricas relacionadas ao processo de desenvolvimento de jogos. Similares as métricas de \textit{performance}, são utilizadas para gerenciar e monitorar métodos que foram adotados, ou que foram utilizados na hora do desenvolvimento do jogo. 
\end{itemize}
\section{Softwares Correlatos}
Nesta parte de documento é levantado os softwares correlacionados, que possuem características ou objetivos parecidos com a plataforma a ser desenvolvida. Os principais são o Steam DB, uma ferramenta que disponibiliza informações sobre o banco de dados da Steam; a Steam Spy, uma plataforma web que disponibiliza informações sobre os jogos e suas vendas por região e o DFC Intelligence, uma ferramenta de pesquisa sobre o mercado de jogos.
\subsection{Steam DB}
Steam DB é uma ferramenta \textit{open source} \textit{third-party} com objetivo de dar um melhor conhecimento sobre os jogos e suas atualizações disponíveis no banco de dados da Steam \cite{steam_db}. Esta ferramenta disponibiliza \textit{rankings} e gráficos de jogos para que o usuário tenha uma melhor visualização. Atualmente o Steam DB apresenta métricas individuais de cada jogo, não oferencendo nenhuma maneira de incremento. Na tabela \ref{table:steam_db} podemos ver as vantagens e desvantagens do Steam DB.
\begin{table} [H]
\centering
\begin{tabular}{|p{7cm}|p{7cm}|}
\hline \textbf{Vantagens} & \textbf{Desvantagens} \\
\hline \textit{Open source} & Não possui política de contribuição \\
\hline \textit{Rankings} e gráficos & Informações individuais de um jogo \\
\hline Gratuito & Precisa de login na Steam \\
\hline Informações individuais de um usuário & \\
\hline
\end{tabular}
\caption{Vantagens e Desvantagens Steam DB}
\label{table:steam_db}
\end{table}
\subsection{Steam Spy}
Steam Spy é uma plataforma Web que a partir de informações sobre os usuários da Steam, disponibiliza informações como o número de donos de determinado jogo ou vendas por região \cite{steam_spy}. Um dos objetivos especificados pelo Steam Spy é o auxílio a desenvolvedores \textit{indies}, porém, para se conseguir todos os dados disponíveis pela plataforma é preciso pagar por eles. Atualmente a Steam Spy para que o usuário possua total acesso aos dados dela, o usuário precisa pagar. Na tabela \ref{table:steam_spy} podemos ver as vantagens e desvantagens do Steam Spy.
\begin{table} [H]
\centering
\begin{tabular}{|p{7cm}|p{7cm}|}
\hline \textbf{Vantagens} & \textbf{Desvantagens} \\
\hline Disponibiliza número de donos e vendas por região & Não é totalmente gratuito \\
\hline Gráficos genéricos e confusos & Não apresenta métricas sobre os jogos \\
\hline Disponibiliza API para seus dados & Não é \textit{open source} \\
\hline
\end{tabular}
\caption{Vantagens e Desvantagens Steam Spy}
\label{table:steam_spy}
\end{table}
\subsection{DFC Intelligence}
DFC Intelligence é uma ferramenta paga que auxiliam desenvolvedoras a tomar conhecimentos sobre estatísticas de seus jogos no mercado. Desenvolvedoras que utilizem essa ferramenta receberão informações sobre o número de vendas, picos de vendas, regiões que mais venderam, entre outras. Também será disponibilizado gráficos e métricas que informam como está o mercado de jogos \cite{dfc_intelligence}. Atualmente a ferramenta DFC Intelligence é focada apenas nos jogos das desenvolvedoras contratantes. Na tabela \ref{table:dfc_intelligence} podemos ver as vantagens e desvantagens do Steam DB.
\begin{table} [H]
\centering
\begin{tabular}{|p{7cm}|p{7cm}|}
\hline \textbf{Vantagens} & \textbf{Desvantagens} \\
\hline Disponibiliza informações sobre o mercado de jogos & Não é gratuita \\
\hline Auxiliam desenvolvedores na hora da criação de um jogo & Não mostrar o mercado de jogos como um todo \\
\hline & Não é \textit{open source} \\
\hline & Foca apenas nos jogos de uma desenvolvedora \\
\hline
\end{tabular}
\caption{Vantagens e Desvantagens DFC Intelligence}
\label{table:dfc_intelligence}
\end{table}