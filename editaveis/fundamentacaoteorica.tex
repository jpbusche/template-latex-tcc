\chapter*[Fundamentação Teórica]{Fundamentação Teórica}
Neste capitulo será abordado a fundamentação teórica para o entendimento do propósito da implementação do projeto. Nele são explicados o que é Game Analytics e Business Intelligence, também serão abordado as posséveis ferramentas que serão utilizadas no decorrer do projeto e também possíveis concorrentes do projeto.
\section*{Análise de Ferramentas}
	\subsection*{Banco de Dados}
		A ferramenta de banco de dados é responsável pelo armazenamento dos dados extraídos dos logs, no contexto desse projeto, este banco deverá armazenar os dados de todos os jogos disponibilizados pela API da Steam.
			\subsubsection*{PostgreSQL}
			Postgre é uma ferramenta open-source de banco de dados que utiliza a linguagem SQL em conjunto com outras funcionalidades que guardam e manipulam os mais complicados dos dados. Postgre é uma ferramenta antiga tendo sua origem datada no ano de 1986[1].

    		Postgre é uma ferramenta que possui muitas funcionalidades com o intuito de ajudar desenvolvedores, administradores para manter a integridade dos dados e criação de sistema com tolerância a falhas.
			\subsubsection*{Elasticsearch}
			Elasticsearch é uma ferramenta open-source, desenvolvida pela Elastic, de análise e busca RESTful capaz de resolver um grande número de casos. É a parte principal de Elastic Stack, servindo como um centro onde armazena os dados[2].

    		Elasticsearch suporta qualquer tipo de dado, além de agregar grande quantidades de dados para se ter uma visão melhor. Entre suas características as que mais se destacam são sua rapidez de busca, capacidade de detecção de falhas, múltiplos tipos de dados e suporte a múltiplas linguagens de programação.
	\subsection*{Extração de Dados}
	A ferramenta de extração de dados é responsável pela extração dos dados dos logs, além de ter que fazer a comunicação com o banco de dados. Essa ferramenta deverá extrair os dados das APIs que serão utilizadas, manipulá-los e inserir no banco de dados.
		\subsubsection*{Logstash}
		Logstash é a ferramenta de extração de dados, desenvolvida pela Elastic, ele faz parte de Elastic Stack em conjunto com o Elasticsearch. Entre as características as que mais destacam são sua capacidade de conseguir extrair qualquer tipo de dado, além de permitir a criação de filtros para transformar e manipular os dados extraídos e também permite uma gama de outputs para onde o dado será enviado[3].
		\subsubsection*{Owner}
		Outra opção de extração de dados e a criação de um software específico para a plataforma. Este software seria criado na linguagem Ruby que também possui ligação com o Elasticsearch. As vantagens desse tipo de software seria que poderia ser implementado um arquitetura de plugins, assim futuramente, outros tipos de dados seriam aceitos na plataforma, além de que caso futuramente seja necessário a mudança do banco de dados, a adaptação para o novo banco seria menos custosa.
	\subsection*{Frequência dos Dados}
	A ferramenta da frequência de dados, será mais um auxiliar na hora da extração de dados. Como será extraídos dados que possuem uma taxa de modificação muito grande, é necessário utilizar uma ferramenta que crie rotinas para essa extração.
		\subsubsection*{Cronjob}
		Cron job é uma ferramenta de agendamento que permite controlar tarefas a serem executadas em tempos pré-configurados. Através dela é possível configurar tarefas automáticas, isso quer dizer que, você pode definir rotinas a serem executadas em um horário específico. Cron jobs são configurados manualmente pelo terminal, podendo serem configurados para atualizações por minutos, horas, dias (do mês ou da semana) e meses
		\subsubsection*{Whenever}
		Whenever é uma gem do ruby, mais especificamente para o framework rails. Está gem utiliza os cron jobs como fundo para gerenciar e controlar tarefas a serem executadas em rotinas ou em um horário específico[4].
		\subsubsection*{Clockwork}
		Clockwork é uma gem do ruby. Está gem também utiliza os cron jobs, porém diferente de whenever, ele não exige que o software possua uma arquitetura MVC. Ele é utilizado especificamente para a criação de rotinas para tarefas que serão executadas automaticamente pela gem[5].